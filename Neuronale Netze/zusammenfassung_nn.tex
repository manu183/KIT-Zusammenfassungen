%%%%%%%%%%%%%%%%%%%%%%%%%%%%%%%%%%%%%%%%%
% Short Sectioned Assignment
% LaTeX Template
% Version 1.0 (5/5/12)
%
% This template has been downloaded from:
% http://www.LaTeXTemplates.com
%
% Original author:
% Frits Wenneker (http://www.howtotex.com)
%
% License:
% CC BY-NC-SA 3.0 (http://creativecommons.org/licenses/by-nc-sa/3.0/)
%
%%%%%%%%%%%%%%%%%%%%%%%%%%%%%%%%%%%%%%%%%

%----------------------------------------------------------------------------------------
%	PACKAGES AND OTHER DOCUMENT CONFIGURATIONS
%----------------------------------------------------------------------------------------

\documentclass[paper=a4, fontsize=11pt]{scrartcl} % A4 paper and 11pt font size

\usepackage[T1]{fontenc} % Use 8-bit encoding that has 256 glyphs
\usepackage[ngerman]{babel}
\usepackage{fourier} % Use the Adobe Utopia font for the document - comment this line to return to the LaTeX default
\usepackage{amsmath,amsfonts,amsthm} % Math packages
\usepackage{graphicx}
\usepackage[utf8]{inputenc}
\usepackage{listings}
\usepackage[section]{placeins}
\usepackage{lipsum} % Used for inserting dummy 'Lorem ipsum' text into the template
\usepackage{float}
\usepackage{multicol}

\usepackage{sectsty} % Allows customizing section commands
\allsectionsfont{\centering \normalfont\scshape} % Make all sections centered, the default font and small caps

\usepackage{fancyhdr} % Custom headers and footers
\pagestyle{fancyplain} % Makes all pages in the document conform to the custom headers and footers
\fancyhead{} % No page header - if you want one, create it in the same way as the footers below
\fancyfoot[L]{} % Empty left footer
\fancyfoot[C]{} % Empty center footer
\fancyfoot[R]{\thepage} % Page numbering for right footer
\renewcommand{\headrulewidth}{0pt} % Remove header underlines
\renewcommand{\footrulewidth}{0pt} % Remove footer underlines
\setlength{\headheight}{13.6pt} % Customize the height of the header

\numberwithin{equation}{section} % Number equations within sections (i.e. 1.1, 1.2, 2.1, 2.2 instead of 1, 2, 3, 4)
\numberwithin{figure}{section} % Number figures within sections (i.e. 1.1, 1.2, 2.1, 2.2 instead of 1, 2, 3, 4)
\numberwithin{table}{section} % Number tables within sections (i.e. 1.1, 1.2, 2.1, 2.2 instead of 1, 2, 3, 4)

\setlength\parindent{0pt} % Removes all indentation from paragraphs - comment this line for an assignment with lots of text

\DeclareMathOperator*{\argmin}{arg\,min}

%----------------------------------------------------------------------------------------
%	TITLE SECTION
%----------------------------------------------------------------------------------------

\newcommand{\horrule}[1]{\rule{\linewidth}{#1}} % Create horizontal rule command with 1 argument of height

\title{	
\normalfont \normalsize 
\textsc{Karlsruhe Institute of Technology} \\ [25pt] % Your university, school and/or department name(s)
\horrule{0.5pt} \\[0.4cm] % Thin top horizontal rule
\huge Neuronale Netze\\ Zusammenfassung SS17 % The assignment title
\horrule{2pt} \\[0.5cm] % Thick bottom horizontal rule
}

\author{Manuel Lang} % Your name

\date{\normalsize\today} % Today's date or a custom date

\begin{document}

%\maketitle % Print the title
%\newpage
%\tableofcontents
%\newpage

\section{Rektifizierung}

\begin{itemize}
\item Nach Rektifizierung verlaufen alle Epipolarlinien horizontal mit derselben v-Koordinate wie der Bildpunkt im anderen Kamerabild. 
\item Nach Korrespondenzen muss nur noch horizontal (in eine Richtung) gesucht werden.
\end{itemize}

\section{Lochkameramodell}

\section{Histogramme}

\section{Faltung}

\section{Komplexes Spektrum zeichnen}

\section{Hough Transformation}

\begin{itemize}
\item Ziel: Erkennung gerader Linien im Bild
\item Definition eines zum Bild dualen Parameterraums: Dimensionalität des Parameterraums=Anzahl Parameter zur Beschreibung des Objekts
\item Ansatz: Stelle Linie durch Normalenvektor (Länge, Winkel) in Polarkoordinaten dar (Sinus-Kosinus-Kurve)
\item $r = x cos \theta + y sin \theta$
\item Realisierung des Parameterraums als Akkumulator Array
\item Für jedes Kantenpixel: Erhöhen der Akkumulatorzellen auf dem zugehörigen Sinusoid
\item Gesuchte Gerade entspricht Maximum im Parameterraum
\item Erweiterung der Hough-Transformation für Geraden auf andere parametrische Formen möglich (Anwendung zur Detektion von Kreisen: $r^2 = x^2 + y^2$)
\end{itemize}

\section{Naive Bayes}

\section{Kamerakoordinaten Geraden}

\section{Morphologische Operatoren}

\section{ZNCC}

\section{SAD}

\section{Bestandteile von Logik}

\section{Fehlerwahrscheinlichkeit (ML)}

\section{Sprachmodelle}

\section{Region Growing}

\begin{enumerate}
\item Wähle Saatpunkt $p_0 = (u_0,v_0)$
\item Initialisiere Region $R = {p_0}$, wähle Schwelle $\epsilon$
\item while $\exists p \in R, q \not\in R$
\end{enumerate}

\section{HSI <> RGB}

\section{Von-Neumann vs NN}

\section{Typen von Hornklauseln}

\section{Rotationsmatrix zu Quaternionen}

\section{Theorie Spracherkennung}

\end{document}